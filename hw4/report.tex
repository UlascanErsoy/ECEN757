\documentclass{article}


% if you need to pass options to natbib, use, e.g.:
%     \PassOptionsToPackage{numbers, compress}{natbib}
% before loading neurips_2023


% ready for submission
\usepackage[final]{neurips_2023}


% to compile a preprint version, e.g., for submission to arXiv, add add the
% [preprint] option:
%     \usepackage[preprint]{neurips_2023}


% to compile a camera-ready version, add the [final] option, e.g.:
%     \usepackage[final]{neurips_2023}


% to avoid loading the natbib package, add option nonatbib:
%    \usepackage[nonatbib]{neurips_2023}

\usepackage[utf8]{inputenc} % allow utf-8 input
\usepackage[T1]{fontenc}    % use 8-bit T1 fonts
\usepackage{hyperref}       % hyperlinks
\usepackage{url}            % simple URL typesetting
\usepackage{booktabs}       % professional-quality tables
\usepackage{amsfonts}       % blackboard math symbols
\usepackage{amsmath}
\usepackage{nicefrac}       % compact symbols for 1/2, etc.
\usepackage{microtype}      % microtypography
\usepackage{xcolor}         % colors
\usepackage{graphicx}
\usepackage{float}
\usepackage{subfig}
\usepackage[most]{tcolorbox}
\usepackage{multicol}

\bibliographystyle{plainnat}


\title{ECEN 757 | Homework 4}


% The \author macro works with any number of authors. There are two commands
% used to separate the names and addresses of multiple authors: \And and \AND.
%
% Using \And between authors leaves it to LaTeX to determine where to break the
% lines. Using \AND forces a line break at that point. So, if LaTeX puts 3 of 4
% authors names on the first line, and the last on the second line, try using
% \AND instead of \And before the third author name.

\author{%
  Muhammed U. Ersoy\\
  M.S. ECEN Student\\
  Texas A\&M University\\
  \texttt{mue@tamu.edu} \\
 }

\begin{document}
\maketitle

\begin{tcolorbox}[colback=blue!5!white,colframe=blue!75!black,title=Problem 16.4]
    The operation create inserts a new bank account at a branch. The transactions T and U
        are defined as follows:
    \begin{verbatim}
        T: aBranch.create("Z");
        U: z.deposit(10); z.deposit(20).
    \end{verbatim}
    Assume that Z does not yet exist. Assume also that the deposit operation does nothing if
    the account given as the argument does not exist. Consider the following interleaving of
    transactions T and U:

    \begin{table}[H]
    \centering
    \begin{tabular}{ll}
    T                                        & U                                   \\ \hline
    \multicolumn{1}{c}{}                   & \multicolumn{1}{c}{z.deposit(10);} \\
    \multicolumn{1}{c}{aBranch.create(Z);} & \multicolumn{1}{c}{}               \\
                                             & z.deposit(20);                      \\ \hline
    \end{tabular}
    \end{table}
    State the balance of Z after their execution in this order. Are these consistent with
    serially equivalent executions of T and U?
\end{tcolorbox}

\begin{tcolorbox}[colback=green!5!white,colframe=green!75!black,title=Answer 16.4]
The balance of $Z$ would be $\$20$. When $U$ first executes deposit on $Z$, the branch doesn't exist thus the operation has no effect.
Then the branch is created by $T$, and $U$ deposits $\$20$.

These are not consistent with a serially equivalent operation. For example if we look at $T \rightarrow U$;

\begin{multicols}{2}
    \begin{table}[H]
    \centering
    \begin{tabular}{ll}
    T                                        & U                                   \\ \hline
    \multicolumn{1}{c}{aBranch.create(Z);}                   & \multicolumn{1}{c}{} \\
    \multicolumn{1}{c}{} & \multicolumn{1}{c}{z.deposit(10);}               \\
                                             & z.deposit(20);                      \\ \hline
    \end{tabular}
    \end{table}

    \begin{table}[H]
    \centering
    \begin{tabular}{ll}
    T                                        & U                                   \\ \hline
    \multicolumn{1}{c}{}                   & \multicolumn{1}{c}{z.deposit(10);} \\
    \multicolumn{1}{c}{} & \multicolumn{1}{c}{z.deposit(20);}               \\
                        aBranch.create(Z);&                      \\ \hline
    \end{tabular}
    \end{table}

\end{multicols}

In the first case the branch is created, and both deposits go through. Our final state would have $\$30$ in the branch. In the other case of $U \rightarrow T$ the branch would not exist
when deposits go through, and the final amount would be $\$0$. Thusfort the transaction outlined is not
serially equivalent.
\end{tcolorbox}

\begin{tcolorbox}[colback=blue!5!white,colframe=blue!75!black,title=Problem 16.9]
    The transactions T and U at the server in Exercise 16.8 are defined as follows:
    \begin{verbatim}
        T: x = read(i); write(j, 44);
        U: write(i, 55); write(j, 66);
    \end{verbatim}
    Initial values of $a_i$ and $a_j$ are 10 and 20, respectively. Which of the following
    interleavings are serially equivalent, and which could occur with two-phase locking?
    \begin{multicols}{2}
        \begin{enumerate}(a)
        \begin{table}[H]
        \begin{tabular}{ll}
                    T                              & U                                \\ \hline
            \multicolumn{1}{l}{x=read(i);} & \multicolumn{1}{l}{}             \\
            \multicolumn{1}{l}{}           & \multicolumn{1}{l}{write(i,55);} \\
            write(j,44);                   &                                  \\
                                           & write(j,66);    
        \end{tabular}
        \end{table}

        \end{enumerate}
        \begin{enumerate}(c)
            \begin{table}[H]
            \begin{tabular}{ll}
            T                                        & U                                   \\ \hline
            \multicolumn{1}{l}{} & \multicolumn{1}{l}{write(i,55);} \\
            \multicolumn{1}{l}{} & \multicolumn{1}{l}{write(j,66);} \\
            x=read(i);           &                                  \\
            write(j,44);         &                      \end{tabular}
            \end{table}
        \end{enumerate}
        \begin{enumerate}(b)
            \begin{table}[H]
            \begin{tabular}{ll}
            T                                        & U                                   \\ \hline
            \multicolumn{1}{l}{x=read(i);}   & \multicolumn{1}{l}{} \\
            \multicolumn{1}{l}{write(j,44);} & \multicolumn{1}{l}{} \\
                                             & write(i,55);         \\
                                             & write(j,66);                    \end{tabular}
            \end{table}
        \end{enumerate}
        \begin{enumerate}(d)
            \begin{table}[H]
            \begin{tabular}{ll}
            T                                        & U                                   \\ \hline
            \multicolumn{1}{l}{}           & \multicolumn{1}{l}{write(i,55);} \\
            \multicolumn{1}{l}{x=read(i);} & \multicolumn{1}{l}{}             \\
                                           & write(j,66);                     \\
            write(j,44);                   &                     \end{tabular}
            \end{table}
        \end{enumerate}

    \end{multicols}
\end{tcolorbox}
\begin{tcolorbox}[colback=green!5!white,colframe=green!75!black,title=Answer 16.9]
(b) and (c) are serially equivalent since in both cases one of the transactions (either $U$ or $T$) accesses all necessary resources before the other one.
\\
(a) and (d) could occur with two-phase locking since all necessary conflicting lock requests are in the same order. 
\end{tcolorbox}

\begin{tcolorbox}[colback=blue!5!white,colframe=blue!75!black,title=Problem 16.10]
Consider a relaxation of two-phase locks in which read-only transactions can release
read locks early. Would a read-only transaction have consistent retrievals? Would the
objects become inconsistent? Illustrate your answer with the following transactions T
and U at the server in Exercise 16.8:
    \begin{verbatim}
        T: x = read (i); y= read(j);
        U: write(i, 55); write(j, 66);
    \end{verbatim}
in which initial values of $a_i$ and $a_j$ are 10 and 20.
\end{tcolorbox}
\begin{tcolorbox}[colback=green!5!white,colframe=green!75!black,title=Answer 16.10]
    The reads could become inconsistent. If the process reads one of the values and releases the lock, 
    another process could write, and if the original process performs a read on another value again, the
    state will become inconsistent.
    \begin{table}[H]
    \centering
    \begin{tabular}{ll}
    T                                        & U                                   \\ \hline
    \multicolumn{1}{l}{x=read(i);}           & \multicolumn{1}{l}{} \\
    \multicolumn{1}{l}{//T releases lock} & \multicolumn{1}{l}{}             \\
    \multicolumn{1}{l}{} & \multicolumn{1}{l}{write(i, 55);}             \\
    \multicolumn{1}{l}{} & \multicolumn{1}{l}{write(j, 66);}             \\
                       y=read(j);            &                     \\
    \end{tabular}
    \end{table}
In this case $T$ read $x$, then released the lock prematurely. $U$ wrote to both $i$, and $j$ because there
was no lock on them. $T$ was not aware of this. Thus the values of $x$ and $y$ ended up being inconsistent.
\end{tcolorbox}
\begin{tcolorbox}[colback=blue!5!white,colframe=blue!75!black,title=Problem 16.16]
Consider optimistic concurrency control as applied to the transactions T and U defined
in Exercise 16.9. Suppose that transactions T and U are active at the same time as one
another. Describe the outcome in each of the following cases:
\begin{enumerate}i) T’s request to commit comes first and backward validation is used.
\end{enumerate}
\begin{enumerate}ii) U’s request to commit comes first and backward validation is used.
\end{enumerate}
\begin{enumerate}iii) T’s request to commit comes first and forward validation is used.
\end{enumerate}
\begin{enumerate}iv) U’s request to commit comes first and forward validation is used.
\end{enumerate}

In each case describe the sequence in which the operations of T and U are performed,
remembering that writes are not carried out until after validation.
\end{tcolorbox}

\begin{tcolorbox}[colback=green!5!white,colframe=green!75!black,title=Answer 16.16]
    \begin{enumerate}i)
        In backwards validation, T's readset is compared with U's write set. Since U has not committed
        by this point all options are valid. 
    \end{enumerate}
    \begin{enumerate}ii)
        In this case U has committed thus the transaction is \textbf not valid.
    \end{enumerate}
    \begin{enumerate}iii)
        T's write set is compared with the reads of with the reads of U. Since U has not reads this is valid.
        Then U's write set would be compared with other active reads. Since T was already committed there are no 
        active requests. Thus this is valid.
    \end{enumerate}
    \begin{enumerate}iv)
        U's write set is compared with T's reads. Since there is an overlap this \textbf fails.
    \end{enumerate}
    
    \textit{Parts of this answer were taken from \citet{Kontogiannis}, I was comparing my original answer and it seemed incomplete so I amended based on this solution manual.}
\end{tcolorbox}

\begin{tcolorbox}[colback=blue!5!white,colframe=blue!75!black,title=Problem 16.19]
        Consider the use of timestamp ordering with each of the example interleavings of
        transactions T and U in Exercise 16.9. Initial values of $a_i$ and $a_j$ are 10 and 20,
        respectively, and initial read and write timestamps are $t_0$. Assume that each transaction
        opens and obtains a timestamp just before its first operation; for example, in (a) T and U
        get timestamps $t_1$ and $t_2$, respectively, where $t_0 < t_1 < t_2.$ Describe in order of increasing
        time the effects of each operation of T and U. For each operation, state the following:
        \begin{enumerate}i) whether the operation may proceed according to the write or read rule;
        \end{enumerate}
        \begin{enumerate}ii) when timestamps are assigned to transactions or objects;
        \end{enumerate}
        \begin{enumerate}iii) when tentative objects are created and when their values are set.
        \end{enumerate}
\end{tcolorbox}

\begin{enumerate}a)\\
        $a_i = 10$ , $T_{write} = \max$, $T_{read} = t_0$\\
        $a_j = 10$ , $T_{write} = \max$, $T_{read} = t_0$\\
        \\
        T: x= read(i);//timestamp = $t_1$\\
        read_rule: $t_1 > T_{write}$ on committed version ($t_0$) and $D_{selected}$ is committed:\\
        $x = 10$ is read, and $\max T_{read} (a_i) = t_1$\\
        \\
        U: write(i,55); //timestamp = $t_2$\\
        write rule: $t_2 \geq \max T_{read} = t_1$ and $t_2$ > write timestamp on committed version.\\
        allows write on tentative version $a_i = 55$. $T_{write} = t_2$\\
        \\
        T: write(j,44);
        write rule: $t_1 \geq \max T_{read} = t_0$ and $t_1$ > write timestamp on committed version = $t_0$\\
        allows tentative write $a_j = 44$; update timestamp of $a_j$ to $t_1$\\
        \\
        U: write(j,66);\\
        write rule: $t_1 \geq \max T_{read} = t_0$ and $t_2$ > write timestamp on committed version = $t_0$\\
        tentative version $a_j = 66$; timestamp = $t_2$\\
        \\
        T commits first $\rightarrow a_j = 44$ $T_{write} = t_1$, $T_{read} = t_0$\\
        U commits second $\rightarrow a_i = 55$ $T_{write} = t_2$, $T_{read} = t_1$\\
        U commits second $\rightarrow a_j = 66$ $T_{write} = t_2$, $T_{read} = t_0$
    \end{enumerate}
    \begin{enumerate}b)\\
        T: x= read(i);//timestamp = $t_1$\\
        read_rule: $t_1 > T_{write}$ on committed version ($t_0$) and $D_{selected}$ is committed:\\
        $x = 10$ is read, and $\max T_{read} (a_i) = t_1$\\
        \\
        T: write(j,44);
        write rule: $t_1 \geq \max T_{read} = t_0$ and $t_1$ > write timestamp on committed version = $t_0$\\
        allows tentative write $a_j = 44$; update timestamp of $a_j$ to $t_1$\\
        \\
        U: write(i,55); //timestamp = $t_2$\\
        write rule: $t_2 \geq \max T_{read} = t_1$ and $t_2$ > write timestamp on committed version = $t_0$\\
        allows write on tentative version $a_i = 55$. $T_{write} = t_2$\\
        \\
        U: write(j,66);\\
        write rule: $t_1 \geq \max T_{read} = t_0$ and $t_2$ > write timestamp on committed version = $t_0$\\
        tentative version $a_j = 66$; timestamp = $t_2$\\
        \\
        T commits first $\rightarrow a_j = 44$ $T_{write} = t_1$, $T_{read} = t_0$\\
        U commits second $\rightarrow a_i = 55$ $T_{write} = t_2$, $T_{read} = t_1$\\
        U commits second $\rightarrow a_j = 66$ $T_{write} = t_2$, $T_{read} = t_0$
    \end{enumerate}

    \begin{enumerate}c)\\
        U: write(i,55); //timestamp = $t_2$\\
        write rule: $t_2 \geq \max T_{read} = t_1$ and $t_2$ > write timestamp on committed version = $t_0$\\
        allows write on tentative version $a_i = 55$. $T_{write} = t_1$\\
        \\
        U: write(j,66);\\
        write rule: $t_1 \geq \max T_{read} = t_0$ and $t_2$ > write timestamp on committed version = $t_0$\\
        tentative version $a_j = 66$; timestamp = $t_1$\\
        \\
        T: x= read(i);//timestamp = $t_2$\\
        read_rule: $t_1 > T_{write}$ on committed version ($t_0$) and $D_{selected}$ is committed:\\
        Since U is not committed yet, T has to wait\\
        \\
        U commits $\rightarrow a_i = 55$ $T_{write} = t_1$, $T_{read} = t_0$\\
        U commits $\rightarrow a_j = 66$ $T_{write} = t_1$, $T_{read} = t_0$\\
        T continues the read and $x = 55$\\
        \\
        T: write(j,44);
        write rule: $t_2 \geq \max T_{read} = t_0$ and $t_1$ > write timestamp on committed version = $t_0$\\
        allows tentative write $a_j = 44$; update timestamp of $a_j$ to $t_2$\\
        \\
        T commits second $\rightarrow a_j = 44$ $T_{write} = t_2$, $\max T_{read} = t_0$\\
        Final state of $a_i = 55$, $T_{write} = t_1$, $\max T_{read} = t_2$\\
    \end{enumerate}

    \begin{enumerate}d)\\
        U: write(i,55); //timestamp = $t_2$\\
        write rule: $t_2 \geq \max T_{read} = t_1$ and $t_2$ > write timestamp on committed version = $t_0$\\
        allows write on tentative version $a_i = 55$. $T_{write} = t_1$\\
        \\
        T: x= read(i);//timestamp = $t_2$\\
        read_rule: $t_1 > T_{write}$ on committed version ($t_0$) and $D_{selected}$ is committed:\\
        Since U is not committed yet, T has to wait\\
        \\
        U: write(j,66);\\
        write rule: $t_1 \geq \max T_{read} = t_0$ and $t_2$ > write timestamp on committed version = $t_0$\\
        tentative version $a_j = 66$; timestamp = $t_1$\\
        \\
        U commits $\rightarrow a_i = 55$ $T_{write} = t_1$, $T_{read} = t_0$\\
        U commits $\rightarrow a_j = 66$ $T_{write} = t_1$, $T_{read} = t_0$\\
        T continues the read and $x = 55$\\
        \\
        T: write(j,44);
        write rule: $t_2 \geq \max T_{read} = t_0$ and $t_1$ > write timestamp on committed version = $t_0$\\
        allows tentative write $a_j = 44$; update timestamp of $a_j$ to $t_2$\\
        \\
        T commits second $\rightarrow a_j = 44$ $T_{write} = t_2$, $\max T_{read} = t_0$\\
        Final state of $a_i = 55$, $T_{write} = t_1$, $\max T_{read} = t_2$\\

    \end{enumerate}
\\
\\
\textit{Parts of 16.19 were taken from \citet{Kontogiannis}, My original answer was incomplete compared to this, I amended it.}

\bibliography{default}
\end{document}


